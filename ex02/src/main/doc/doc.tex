\documentclass[a4paper,headings=small]{scrartcl}
\KOMAoptions{DIV=12}

\usepackage[utf8x]{inputenc}
\usepackage{amsmath}
\usepackage{graphicx}
\usepackage{subfigure}
\usepackage{multirow}
\usepackage{listings}
\usepackage{subfigure}

% define style of numbering
\numberwithin{equation}{section} % use separate numbering per section
\numberwithin{figure}{section}   % use separate numbering per section

% instead of using indents to denote a new paragraph, we add space before it
\setlength{\parindent}{0pt}
\setlength{\parskip}{10pt plus 1pt minus 1pt}


\title{Noise Suppression}
\subtitle{Excercise 2 \\ Digital Image Processing - SS13}
\author{\textbf{Group F}: Aouatef Ouerhani (312955), Frank Ng (316500),\\ Robin Vobruba (343773)}
\date{\today}

\pdfinfo{%
  /Title    (Digital Image Processing - SS13 - Excercise 2 - Noise Suppression)
  /Author   (Group F: Aouatef Ouerhani (312955), Frank Ng (316500), Robin Vobruba (343773))
  /Creator  ()
  /Producer ()
  /Subject  ()
  /Keywords ()
  %Version 1
}

% Simple picture reference
%   Usage: \image{#1}{#2}{#3}
%     #1: file-name of the image
%     #2: percentual width (decimal)
%     #3: caption/description
%
%   Example:
%     \image{myPicture}{0.8}{My huge house}
%     See fig. \ref{fig:myPicture}.
\newcommand{\image}[3]{
	\begin{figure}[htbp]
		\centering
		\includegraphics[width=#2\textwidth]{#1}
		\caption{#3}
		\label{fig:#1}
	\end{figure}
}
\newcommand{\imgRoot}{../resources/img}
\newcommand{\imgGeneratedRoot}{../../../target}

\begin{document}

\maketitle

\section{Practical}

See lst. \ref{lst:theCode}.


\section{Theoretical}

See the original image in fig. \ref{fig:original}.

\begin{figure}[htbp]
	\centering
	\includegraphics[width=0.4\textwidth]{\imgGeneratedRoot/original.jpg}
	\caption{original image}
	\label{fig:original}
\end{figure}

\begin{figure}[ht]
	\centering

	\subfigure[shot noise]{
		\includegraphics[width=0.4\textwidth]{\imgGeneratedRoot/noiseType_1.jpg}
		\label{fig:noiseTypeShot}
	}
	\subfigure[shot noise restored with median filter ($k = 3$)]{
		\includegraphics[width=0.4\textwidth]{\imgGeneratedRoot/restorated1.jpg}
		\label{fig:restored1}
	}

	\subfigure[Gaussian noise]{
		\includegraphics[width=0.4\textwidth]{\imgGeneratedRoot/noiseType_2.jpg}
		\label{fig:noiseTypeGauss}
	}
	\subfigure[Gaussian noise restored with moving average filter ($k = 15$, $\text{threshold} = 30.0$)]{
		\includegraphics[width=0.4\textwidth]{\imgGeneratedRoot/restorated2.jpg}
		\label{fig:restored2}
	}

	\caption[Optional caption for list of figures]{
		noisy \& de-noised images \subref{fig:noiseTypeShot}, \subref{fig:noiseTypeGauss}
	}
	\label{fig:noiseAndRestored}
\end{figure}


\subsection{1. When should the median filter be applied to an image and when the moving average filter?}

The median filter should be used with shot noise,
like in fig. \ref{fig:noiseTypeShot}
(some pixels have correct values, some have heavily wrong values).
We can get a result like in fig. \ref{fig:restored1}.

The moving average filter works well for Gaussian noise,
like in fig. \ref{fig:noiseTypeGauss}
(each pixel deviates somewhat from its correct value, some more, some less).
We can get a result like in fig. \ref{fig:restored2}.


\subsection{2. Explain your answer to question 1.}

The median filter is a non-linear filter and belongs to the class of edge preserving, smoothing filters.
This filter is a very effective filter to remove shot noise,
because it would rank strong spike-like isolated values in the first or last places.
The median filter takes the median value.
Therefore, shot noises which have very high or very low values would not be used, but be eliminated.

On the contrary, the moving average filter would factor spike-like isolated values into the process.
A few very high or very low values would change the result to a non acceptable one.

For Gaussian noise, it is better to take the moving average filter.
The median filter removes isolated noise pixels, but it does not smooth detail as much as the moving average filter.
Smoothing blurs the noise and detail, which makes sense for Gaussian noise.

\subsection{3. Is there a general better choice than the moving average filter?}

The Gaussian filter is a general better choice than the moving average filter.

\subsection{4. Explain your answer to question 3.}

In general, the Gaussian filter is the ideal smoothing filter.
It is similar to the average mean filter,
but it uses a different kernel.
The Gaussian filter outputs a weighted average of the pixel's neighbourhood,
with the average weighted more towards the value of the central pixels.
Because of this, the result of the Gaussian filter is a gentler smoothing
and a better preserving of the edges than the moving average filter.


\newpage
\section{Source Code:}

\lstinputlisting[breaklines=true,label=lst:theCode,caption=The code]{../native/dip2.cpp}

\end{document}
