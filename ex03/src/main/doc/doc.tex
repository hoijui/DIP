\documentclass[a4paper,headings=small]{scrartcl}
\KOMAoptions{DIV=12}

\usepackage[utf8x]{inputenc}
\usepackage{amsmath}
\usepackage{graphicx}
\usepackage{subfigure}
\usepackage{multirow}
\usepackage{listings}
\usepackage{subfigure}

% define style of numbering
\numberwithin{equation}{section} % use separate numbering per section
\numberwithin{figure}{section}   % use separate numbering per section

% instead of using indents to denote a new paragraph, we add space before it
\setlength{\parindent}{0pt}
\setlength{\parskip}{10pt plus 1pt minus 1pt}


\title{ Unsharp Masking with Frequency Convolution}
\subtitle{Excercise 3 \\ Digital Image Processing - SS13}
\author{\textbf{Group F}: Aouatef Ouerhani (312955), Frank Ng (316500),\\ Robin Vobruba (343773)}
\date{\today}

\pdfinfo{%
  /Title    (Digital Image Processing - SS13 - Excercise 3 - Unsharp Masking with Frequency Convolution)
  /Author   (Group F: Aouatef Ouerhani (312955), Frank Ng (316500), Robin Vobruba (343773))
  /Creator  ()
  /Producer ()
  /Subject  ()
  /Keywords ()
  %Version 1
}

\newcommand{\imgRoot}{../resources/img}
\newcommand{\imgGeneratedRoot}{../../../target}

\begin{document}

\maketitle


\section{Quality of the result}

See the input in fig. \ref{fig:input},
the degraded version in fig. \ref{fig:degraded},
the USM difference to the degraded image in fig. \ref{fig:difference},
and the USM-enhanced image in fig. \ref{fig:enhanced},


\begin{figure}[ht]
	\centering

	\subfigure[input]{
		\includegraphics[width=0.45\textwidth]{\imgGeneratedRoot/input.jpg}
		\label{fig:input}
	}
	\subfigure[degraded]{
		\includegraphics[width=0.45\textwidth]{\imgGeneratedRoot/dedgraded.png}
		\label{fig:degraded}
	}

	\subfigure[difference (kernel size: 25 x 25)]{
		\includegraphics[width=0.45\textwidth]{\imgGeneratedRoot/input_USM_25x25_frequencyDomain_diff2original.png}
		\label{fig:difference}
	}
	\subfigure[enhanced (kernel size: 25 x 25)]{
		\includegraphics[width=0.45\textwidth]{\imgGeneratedRoot/input_USM_25x25_frequencyDomain_enhanced.png}
		\label{fig:enhanced}
	}

% 	\caption[Optional caption for list of figures]{
% 		noisy \& de-noised images \subref{fig:noiseTypeShot}, \subref{fig:noiseTypeGauss}
% 	}
	\label{fig:imagesQuality}
\end{figure}


\section{Algorithmic performance}

For convolution directly in the spacial domain,
computation time grows linearly with the number of image pixels,
and with the number of kernel pixels,
respectively squared with the kernel size (side length),
as can be seen in \ref{fig:plotSpacial}.

For convolution through conversion into and back from the frequency domain,
computation time also grows linearly with the number of image pixels,
put stays constant for different kernel sizes,
as can be seen in fig. \ref{fig:plotFrequency}.
This makes sense, because we always first convert the kernel into an image
of the size of the to-be-processed image (independent of the kernel size).
From this point onwards, all calculations
(DFT, multiplication of imageF and kernelF, IDFT)
are kernel size independent.

The raw experiment data for both algorithms can be seen in
fig. \ref{fig:dataSpacial} and fig. \ref{fig:dataFrequency}.

\begin{figure}[ht]
	\includegraphics[width=0.9\textwidth]{\imgRoot/spacialDomainPlot.png}
	\label{fig:plotSpacial}
\end{figure}

\begin{figure}[ht]
	\includegraphics[width=0.9\textwidth]{\imgRoot/frequencyDomainPlot.png}
	\label{fig:plotFrequency}
\end{figure}

\begin{figure}[ht]
	\centering

	\subfigure[spacial domain data]{
		\includegraphics[width=0.45\textwidth]{\imgRoot/spacialDomainData.png}
		\label{fig:dataSpacial}
	}
	\subfigure[frequency domain data]{
		\includegraphics[width=0.414\textwidth]{\imgRoot/frequencyDomainData.png}
		\label{fig:dataFrequency}
	}
	\label{fig:data}
\end{figure}

\newpage
\section{Source Code:}

\lstinputlisting[breaklines=true,label=lst:theCode,caption=The code]{../native/dip3.cpp}

\end{document}
