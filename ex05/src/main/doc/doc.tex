\documentclass[a4paper,headings=small]{scrartcl}
\KOMAoptions{DIV=12}

\usepackage[utf8x]{inputenc}
\usepackage{amsmath}
\usepackage{graphicx}
\usepackage{subfigure}
\usepackage{multirow}
\usepackage{listings}
\usepackage{subfigure}

% define style of numbering
\numberwithin{equation}{section} % use separate numbering per section
\numberwithin{figure}{section}   % use separate numbering per section

% instead of using indents to denote a new paragraph, we add space before it
\setlength{\parindent}{0pt}
\setlength{\parskip}{10pt plus 1pt minus 1pt}


\title{Interest Points}
\subtitle{Excercise 5 \\ Digital Image Processing - SS13}
\author{\textbf{Group F}: Aouatef Ouerhani (312955), Frank Ng (316500),\\ Robin Vobruba (343773)}
\date{\today}

\pdfinfo{%
  /Title    (Digital Image Processing - SS13 - Excercise 5 - Interest Points)
  /Author   (Group F: Aouatef Ouerhani (312955), Frank Ng (316500), Robin Vobruba (343773))
  /Creator  ()
  /Producer ()
  /Subject  ()
  /Keywords ()
  %Version 1
}

\newcommand{\imgRoot}{../resources/img}
\newcommand{\imgGeneratedRoot}{../../../target}

\begin{document}

\maketitle

See the result in fig. \ref{fig:keypoints}.
Even though it is suggested to use a $w_{\text{min}}$ of 0.5 till 1.5,
we had to use a value of 30.0 or higher, to get usable results.

\begin{figure}[ht]
	\centering
	\includegraphics[width=0.7\textwidth]{\imgGeneratedRoot/keypoints.png}
	\caption{
		The key-points found with $w_{\text{min}} = 60.0$ and $q_{\text{min}} = 0.6$.
	}
	\label{fig:keypoints}
\end{figure}


\newpage
\section{Source Code:}

\lstinputlisting[breaklines=true,label=lst:theCode,caption=The code]{../native/dip5.cpp}

\end{document}
