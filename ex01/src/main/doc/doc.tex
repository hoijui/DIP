\documentclass[a4paper,headings=small]{scrartcl}
\KOMAoptions{DIV=12}

\usepackage[utf8x]{inputenc}
\usepackage{amsmath}
\usepackage{graphicx}
\usepackage{multirow}
\usepackage{listings}
\usepackage{subfigure}

% define style of numbering
\numberwithin{equation}{section} % use separate numbering per section
\numberwithin{figure}{section}   % use separate numbering per section

% instead of using indents to denote a new paragraph, we add space before it
\setlength{\parindent}{0pt}
\setlength{\parskip}{10pt plus 1pt minus 1pt}

\title{C++ and OpenCV}
\subtitle{Excercise 1 \\ Digital Image Processing - SS13}
\author{\textbf{Group F}: Aouatef Ouerhani (312955), Frank Ng (316500),\\ Robin Vobruba (343773), Lucas van Walstijn}
\date{\today}

\pdfinfo{%
  /Title    (Digital Image Processing - SS13 - Excercise 1 - C++ and OpenCV)
  /Author   (Group F: Aouatef Ouerhani (312955), Frank Ng (316500), Robin Vobruba (343773), Lucas van Walstijn)
  /Creator  ()
  /Producer ()
  /Subject  ()
  /Keywords ()
  %Version 1
}

% Simple picture reference
%   Usage: \image{#1}{#2}{#3}
%     #1: file-name of the image
%     #2: percentual width (decimal)
%     #3: caption/description
%
%   Example:
%     \image{myPicture}{0.8}{My huge house}
%     See fig. \ref{fig:myPicture}.
\newcommand{\image}[3]{
	\begin{figure}[htbp]
		\centering
		\includegraphics[width=#2\textwidth]{#1}
		\caption{#3}
		\label{fig:#1}
	\end{figure}
}
\newcommand{\imgRoot}{../resources/img}
\newcommand{\imgGeneratedRoot}{../../../target}

\begin{document}

\maketitle

\section{Color Reduction}

The input- and output-images can be see in
fig. \ref{fig:inOutImg}).

\begin{figure}[htbp]
	\centering
	\includegraphics[width=0.4\textwidth]{\imgRoot/input.jpg}
	\includegraphics[width=0.4\textwidth]{\imgGeneratedRoot/result.jpg}
	\caption{The original- and the color-reduced-image}
	\label{fig:inOutImg}
\end{figure}


\newpage
\section{Source Code:}

\begin{lstlisting}[label=interstingFunction,caption=The function with the color-reduction code.]
Mat doSomethingThatMyTutorIsGonnaLike(Mat& img) {

	int div = 64;
	int nl = img.rows;
	int nc = img.cols * img.channels();

	for (int j = 0; j < nl; j++) {
		uchar* data = img.ptr<uchar>(j);
		for (int i=0; i<nc; i++) {
			data[i] = (data[i] / div * div) + (div / 2);
		}
	}

	return img;
}
\end{lstlisting}

\end{document}
